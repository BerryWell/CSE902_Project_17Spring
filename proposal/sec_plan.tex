%!TEX root = main.tex

In this project, we aim to develop and implement a deep learning algorithm that takes a fingerprint image as an input and classify it into one of the five pattern class types of a) Arch; b) Tented Arch; c) Left Loop; d) Right Loop; or e) Whorl. 

\subsection{Feature Extraction}
%
We will first apply raw fingerprint images to train a CNN for classification. The outputs of some intermediate layer of CNN will be used as features for possibly a support vector machine classifier.

For CNN architecture, we will first use canonical architecture ( such as 5 \textit{convolutional} + 3 \textit{fully-connected} in \textit{AlexNet}\cite{krizhevsky2012imagenet}).
%
We will then modify the CNN architecture to improve the performance.
%
\subsection{Classifier}
%
We will consider two classifiers. The first one is the prediction layer of CNN. The values in last layer indicates the predicted probabilities of each class.
%
The second one is support vector machine (SVM whose input features comprise of the CNN’s middle or last layers.

\subsection{Data Augmentation}

%
To further improve the performance, we will use data augmentation technique to generate more training samples in order to increase the generalization ability of our model. 
%
The augmentation methods include image rotation, resizing and translation.

\subsection{Multi-Task Learning}

Multi-Task Learning (MTL)\cite{caruana1998multitask} aims to improve the performance of multiple classification tasks by learning them jointly.
%
In MTL, some tasks can benefit from auxiliary information which is introduced by other tasks and the performance is improved\cite{zhang2016learning}.
%
In our project, the network is trained to perform both the primary task (fingerprint type classification)and one or more auxiliary tasks. 
%
The classification task is expected to benefit from the auxiliary task by jointly training them together.
%
This process can also be view as incorporating human knowledge into to training procedure.

To obtain labels for auxiliary tasks, we will use existing methods. For example, orientation flow estimation\cite{NFIQ}.
