
The challenge of classifying fingerprint includes: 
%
1) the quality of some fingerprints images are poor; 
%
2) the inter-class dissimilarity is small and the intra-class similarity is small; 
%
3) There are ambiguities in some labels. Some fingerprints can be classified into multiple classes, or different classes by different fingerprint experts.

To solve these problems, many researchers propose to use handcrafted features instead of raw fingerprint images for classification and many methods have been proposed, including ridge, orientation field, singular point.
%
Kai Cao \textit{et al.}\cite{cao2013fingerprint} propose a novel method to extract fingerprint orientation feature and use a hierarchical classifier for classification.
%
Ruxin Wang \textit{et al.} \cite{wang2014fingerprint} also use orientation filed as features. By adopting a stacked autoencoder , they achieve 93.1\% in four-class classification.
%

Using accurate handcrafted features can improve performance.  However, due to the existence of noise and poor image quality, the accuracy of handcrafted features cannot be guaranteed.
%
Convolutional neural network (CNN) has the capability of learning features and it can be directly applied on raw images. CNN also exhibits powerful classification capability in many areas\cite{lecun2015deep}\cite{szegedy2016rethinking}.


